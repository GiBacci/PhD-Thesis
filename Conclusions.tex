%%%%%%%%%%%%%%%%%%%%%%%%%%%%%%%%%%%%%%%%%%%%%%
\logvartrue
\chapter{General Conclusion and Future Perspectives}
%%%%%%%%%%%%%%%%%%%%%%%%%%%%%%%%%%%%%%%%%%%%%%
The inclusion of the suffix ``meta'' in the word genomic is a tale of two different fields of microbiology which share a common background: clinical microbiology and environmental microbiology. With the development of the small subunit RNA phylotype technique and its application to environmental data, microbiologists have been able to explore bacterial communities in different environments making inference on the dynamics between bacterial cells sharing the same habitat. In addition, several research efforts have been focused on characterizing microbial populations by obtaining estimates of the number and the taxonomy of individuals within the inspected community. Temporal changes in a population are able to inform us about the growth rate of that population, whereas geographic or spatial changes tell us about the existing connections between the environment and the bacterial species living in it. The analysis of this ``universally conserved'' gene has transformed microbial ecology from a primarily descriptive discipline to a fully-fledged quantitative ecology discipline. On the other hand, the metagenomic approach has allowed clinical microbiologists to overcome the limitations imposed by the standard microbiological methods based on the isolation and the successive cultivation of microorganisms. Therefore, the ``meta'' approach has let to the detection of novel bacterial strains able to cause infection diseases or to worsen patients conditions. The assembly of whole bacterial genomes directly from patient samples has provided the high-resolution details necessary to understanding the genetic basis of bacterial pathogenesis, its evolutionary foundations, and the development of new therapeutic treatments for disease. Despite these conceptual changes in microbiology, it is not worth noticing that the application of metagenomic approaches has been fuelled by substantial technological advancements in DNA sequencing and analysis of produced data. But all these technological and conceptual advances, have introduced new problems related to the analysis of metagenomic data. Indeed, the increasing number of sequences generated from metagenomic studies is often difficult to handle using the existing tools, not specifically designed for this purpose (e.g. the available database searching algorithms are very slow if used for metagenomic data). In addition, analysing complex bacterial communities remains a challenge because of both experimental and data analysis problems. The development of methods and software explicitly designed to cope with all the problems related to metagenomic studies is becoming mandatory for making a step forward in this novel discipline. In this perspective, the interest in bioinformatics as a separate field of science is increasing every year. Therefore, bioinformatics is becoming an interdisciplinary field aiming at the development of methods and software tools for understanding biological data. This discipline combines computer science, statistics, mathematics, and engineering to study and process data obtained from biological experiments. Developing bioinformatic software that can be easily used even by wet-lab microbiologists will be an inevitable stage in the metagenomic perspective. Providing bioinformatic tools either with web-interfaces or with graphical user interfaces will help the diffusion of these instruments making possible for microbiologists to analyse their own data without the need to know a programming language. In conclusion, in the last decade metagenomics has become a very important field of microbiology both from an environmental and from a clinical point of view. Despite its youth, several efforts have been already taken to develop methods and instruments specifically designed for this discipline. However, from a bioinformatic context metagenomics studies are still difficult to handle resulting in complex and less standardized pipelines. In the future, we must focus our attention on the fine tuning of these methods making this novel field of biology accessible to everyone.\\
