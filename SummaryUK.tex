%*******************************************************
% Abstract
%*******************************************************
\pdfbookmark[3]{Abstract}{abstract}
\chapter{Abstract}
Bacteria are everywhere and they make up for their microscopic dimension in sheer number. Thanks to their great functional and taxonomic variability, bacteria plays a pivotal role in several environments. In soil they are responsible for some of the most important biogeochemical cycles and for the cycling of organic compounds. They reside even on both the surface and in deep layers of human skin, in the saliva and oral mucosa, in the conjunctiva, and in the gastrointestinal tracts performing tasks that can be useful for the human host. They were able to colonize extreme environments and, thanks to their ability to break down hazardous substances into less toxic or non toxic substances, they are used as bioremediation agents for removing pollutant from contaminated sites. Nevertheless, the study of microbial diversity patterns is still hampered by the enormous diversity of microbial communities and the lack of resources to sample them exhaustively.\\
Difficulties in bacterial cultivation and isolation are even today one of the main obstacle for the correct identification of both bacterial taxonomies and bacterial functions. A range of novel methods, most of which are based on rRNA and rDNA analyses, have uncovered part of the microbial diversity in several environments but the percentage of characterized organisms remains low (approximately 1-5\% of the whole bacterial biodiversity). The next step in the ``genomic era'' is to extract genomic, taxonomic and functional informations directly from bacterial communities genome (the metagenome). This concept moved our perspective of bacterial diversity to a ``meta'' point of view. To this end, the advent of the Next Generation Sequencing (NGS) techniques has made it technically possible to exhaustively sequence entire bacterial communities without the need to isolate single strains. Therefore, developing new methods and protocols for studying bacterial patterns in different environments is becoming one of the most competitive challenge to date.\\
The work here presented is part of this perspective and it can be easily divided into three different major parts: the first (named ``Small tools for big data") deals with all issues related to processing and exploitation of sequence data in a metagenomic analysis. In particular, performing this kind of analysis require generating a huge amount of data which can't be easily managed even by most sophisticated computers. For this reason is crucial developing bioinformatic algorithms able to handle a big number of sequences with a small amount of memory and processor resources. Results obtained within this part of the work have underlined the need both to improve available tools and to develop new software for analysing metagenomic data. Specifically, one of the most used tool for taxonomic classification of rRNA sequences from metabarcoding analyses was benchmarked: the ``Ribosomal Database Project'' Classifier (RDP Classifier). This analysis has shown a decreasing level of resolution passing from Phylum to Genus level highlighting the need of continuous improvements in current classification algorithms. On the other hand, a new tool for quality refinement of metagenomic data has been developed. This tool is able to remove low quality segments from raw sequence data retaining as much information as possible. Furthermore, the amount of memory required for this tool to work is proportional to the length of the sequence analysed and it is not related to the number of sequences to analyse, making this tool very useful for metagenomics analyses.\\
The second part of this work includes three different chapters named: ``Bacteria-environment interaction'', ``A life in transition: bacteria in harsh environments'' and ``Bacteria in agro-environments''. Basically, this part covers both functional and taxonomic classification of bacteria in different environments. In particular, changes in bacterial community composition have been inspected from three different perspectives: natural environments, extreme environments and agricultural environments. In all these ambiances, bacteria were found capable of shaping their composition based on either external perturbing agents (entropic gradient, salinity, pollution...) or time. Developing methods for studying bacterial community is one of the crucial tasks in metagenomics where even obtain an overview of bacterial community data could be a very difficult task (especially in complex environment like soil).\\
In the last part of my work, bacterial communities have been analysed as both human and animal hosts. From the animal side, the gut microbiota of talitrid amphipods has been analysed in order to inspect changes in bacterial communities related to different species and locations. Results obtained have shown that is possible to generate a bacterial fingerprint for both inspected factors, underlining that, either different species or different diets are able to influence bacterial gut composition. From the human side, the composition of bacterial communities has been inspected in patients with a specific pathological conditions: cystic fibrosis. Understanding differences in bacterial populations related to this pathological state is very important for either developing new treatments or improving existing ones. In addition, reported results highlight the importance of microbiome inspection for the selection of bacterial groups that may be related to the uprising of a particular human disease.\\