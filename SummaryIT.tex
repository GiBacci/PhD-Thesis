%*******************************************************
% Sommario
%*******************************************************
\pdfbookmark[4]{Sommario}{sommario}
\chapter{Sommario}
\begin{otherlanguage}{italian}

\small
La recente rivoluzione nelle tecnologie di sequenziamento ha portato ad un cambiamento impressionante nel campo della biologia, specialmente per quanto riguarda la genetica e la microbiologia: analisi complesse come il sequenziamento completo di un genoma batterico sono divenute economicamente possibili e facilmente eseguibili; grazie a questo cambiamento, sono nate discipline come la genomica, la fenomica e la genomica comparativa. Studi microbiologici non sono più limitati ad un singolo ceppo di una particolare specie, ma possono essere applicate ad un vasto numero di ceppi, creando la possibilità di studiare la diversità naturale all'interno di una specie batterica, la quale può avere un notevole impatto nelle molte applicazioni della microbiologia. Una di queste applicazioni riguarda l'agricoltura e in modo specifico l'interazione pianta-batterio, dove le specie batteriche giocano un ruolo importante nella sostenibilità agricola e nel ciclo dell'azoto, attraverso la simbiosi rizobica.

Questa tesi è dunque focalizzata sull'analisi dei determinanti genetici della variabilità fenotipica naturale all'interno della simbiosi della specie \textit{Sinorhizobium meliloti} con le piante della specie \textit{Medicago}: in particolare sulle differenze nella capacità di promuovere la crescita della pianta e sulla resistenza agli stress ambientali. L'obbiettivo a lungo termine di questo tipo di ricerca è quello di definire una serie di elementi genetici per il miglioramento dell'applicazione agricola di questo batterio, il quale ha un impatto di tipo economico, sociale ed ambientale.

La prima parte della tesi è incentrata sullo sviluppo di metodi computazionali per l'analisi dell'enorme mole di dati ottenibili dalle nuove tecnologie afferenti all'era \textit{omica}: in particolare è stato sviluppato un software per il miglioramento dei genomi batterici incompleti (CONTIGuator) e una \textit{suite} per l'analisi combinata di esperimenti genomici e fenomici (DuctApe). Per entrambi i software sono presentati applicazioni su dati sperimentali reali.

La seconda parte della tesi si occupa di individuare i determinanti genetici dell'interazione pianta-batterio all'interno della classe \textit{Alphaproteobacteria}, dove si trovano la maggior parte delle specie che interagiscono con le piante. L'intento è quello di capire se esista un insieme condiviso di elementi genetici per questo tipo di interazione, e quanto questo sia condiviso negli altri \textit{taxa}; una serie di elementi comuni sono in effetti stati trovati, specialmente per l'interazione simbiotica.

L'ultima parte della tesi riguarda una delle specie analizzate nella parte precedente (\textit{S. meliloti}); attraverso approcci di genomica comparata è stata studiata la varibilità naturale nel fenotipo di promozione di crescita della pianta e una serie di fattori determinanti è stata individuata, comprendenti pattern di presenza/assenza di geni simbiotici chiave, così come elementi regolativi. Inoltre è stata presa in considerazione anche l'evoluzione funzionale della peculiare struttura genomica di \textit{S. meliloti} per meglio capire la sua origine ed evoluzione, arrivando a definire una visione più generale sull'evoluzione dei genomi rizobici e di altri genomi batterici complessi.

Nel complesso questa tesi mostra come sia possibile usare l'enorme quantità di dati resa disponibile dalle tecnologie \textit{omiche} per unire la variabilità genomica con quella fenotipica; in particolare la variabilità nella simbiosi pianta-batterio ed il suo impatto sull'agricoltura. Sia gli strumenti computazionali che l'approccio usato in questa tesi sono applicabili ad altre specie con fenotipi complessi e varibili, i quali sono uno dei fattori chiave per un'agricoltura sostenibile.

\end{otherlanguage}