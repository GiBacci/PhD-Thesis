%*******************************************************
% Sommario
%*******************************************************
\pdfbookmark[4]{Sommario}{sommario}
\chapter{Sommario}
\begin{otherlanguage}{italian}

I batteri sono capaci di colonizzare quasi tutti gli ambienti presenti sul pianeta terra. Infatti, grazie alla loro plasticità funzionale e tassonomica, sono in grado di giocare un roulo importante in molti ambienti diversi. Nel suolo sono tra i principali responsabili del ciclo dei compositi organici ed altri importanti cicli biogeochimici ma riescono a vivere anche all'interno di organismi più grandi come ad esempio l'uomo. Infatti intere comunità batteriche sono state caratterizzare in diversi distretti del corpo umano come: sulla pelle, nella saliva, nella mucosa orale, ma anche nella congiuntiva e nel tratto gastrointestinale svolgendo funzioni essenziali per lo sviluppo dell'ospite. La loro plasticità li rende in grado di crescere e colonizzare anche gli ambienti più estremi e, grazie alla loro capacità di trasformare sostanze pericolose in sostanze poco (o per niente) tossiche, sono utilizzati anche come agenti per la bioremediation. Purtroppo però lo studio delle popolazioni microbiche rimane difficoltoso sia per la loro grande variabilità che per la mancanza di metodi per campionare esaustivamente complesse popolazioni.\\
Uno dei maggiori ostacoli all'identificazione di specie batteriche è la coltivazione e l'isolamento delle specie di interesse in condizioni di laboratorio. Infatti è stimato che solo l'1\% dei batteri sia in grado di essere coltivato e studiato usando metodi microbioloigici standard. Metodi di analisi innovativi, la maggiorn parte dei quali basati sull'analisi della sequenza del DNA ribosomiale, hanno permesso di scoprire nuove specie e funzioni batteriche ma la percentuale dei microorganismi caratterizzati rimane bassa (circa 1-5\% della biodiversità batterca totale). Il prossimo passo nell'era genomica sarà l'estrazione delle informazioni tassonomiche e funzionali direttamente dal genoma delle comunità batteriche (il cosiddetto ``metagenoma''). Questa sfida sta spostando la concezione della diversità batterica verso una prospettiva ``meta''. Per questo l'avvento delle tecniche di sequenziamento di nuova generazione (NGS) ha reso tecnicamente possibile il sequenziamento di intere comunità batteriche senza il bisogno di isolare singole specie. Lo sviluppo di nuovi metodi e protocolli per lo studio dei pattern batterici in ambienti diversi sta diventando una delle sfide più difficili dei nostri tempi.\\

\end{otherlanguage}