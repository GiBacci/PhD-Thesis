%*******************************************************
% Sommario
%*******************************************************
\pdfbookmark[4]{Sommario}{sommario}
\chapter{Sommario}
\begin{otherlanguage}{italian}

I batteri sono capaci di colonizzare quasi tutti gli ambienti presenti sul pianeta terra. Infatti, grazie alla loro plasticità funzionale e tassonomica, sono giocano un roulo importante in molti ambienti diversi. Nel suolo sono tra i principali responsabili del ciclo dei compositi organici ed altri importanti cicli biogeochimici ma riescono anche a vivere all'interno di organismi più grandi come ad esempio l'uomo. Infatti intere comunità batteriche sono state caratterizzare in diversi distretti del corpo umano come: sulla pelle, nella saliva, nella mucosa orale, nella congiuntiva e nel tratto gastrointestinale svolgendo funzioni essenziali per lo sviluppo dell'ospite. La loro plasticità li rende in grado di crescere e colonizzare anche gli ambienti più estremi e, grazie alla loro capacità di trasformare sostanze pericolose in sostanze poco (o per niente) tossiche, sono spesso utilizzati come agenti per la bonifica di aree inquinate. Purtroppo però lo studio delle popolazioni microbiche rimane difficoltoso sia a causa della loro grande variabilità genomica che per la mancanza di metodi per campionare esaustivamente complesse popolazioni.\\
Uno dei maggiori ostacoli all'identificazione di specie batteriche è la coltivazione e l'isolamento delle specie di interesse in condizioni di laboratorio. Infatti è stimato che solo l'1\% dei batteri sia in grado di essere coltivato e studiato usando metodi microbioloigici standard. Metodi di analisi innovativi, la maggiorn parte dei quali basati sull'analisi della sequenza del DNA ribosomiale, hanno permesso di scoprire nuove specie e funzioni batteriche ma la percentuale dei microorganismi caratterizzati rimane bassa (circa 1-5\% della biodiversità batterca totale). Il prossimo passo nell'era genomica sarà l'estrazione delle informazioni tassonomiche e funzionali direttamente dal genoma delle comunità batteriche (il cosiddetto ``metagenoma''). Questa sfida sta spostando la concezione della diversità batterica verso una prospettiva ``meta'' cioè un aprospettiva in grado di ``trascendere'' la definizione tradizionale di specie. Infatti, l'avvento delle tecniche di sequenziamento di nuova generazione (NGS) ha reso tecnicamente possibile il sequenziamento di intere comunità batteriche senza il bisogno di isolare singole specie rendendo lo sviluppo di nuovi metodi e protocolli per lo studio dei pattern batterici in ambienti diversi come una delle sfide più difficili e competitive dei nostri tempi.\\
Il lavoro qui presentato si inserisce in questo contesto e può essere concettualmente diviso in tre grandi aree tematiche. La prima area (comprendete il capitolo chiamato ``Small tools for big data'')  affronta i problemi legati al pre-processamento delle sequenze metagenomiche grezze e all'analisi finale delle sequenze ottenute. Infatti, affrontare uno studio metagenomico richiede solitamente la produzione di un elevato numero di sequenze rendendo questo tipo di analisi molto impegnativo da un punto di vista computazionale. Per questa ragione, la produzione di algoritmi bioinformatici in grado di analizzare grandi quantità di dati con una piccola quantità di risorse informatiche, sta divenatndo fondamentale per l'analisi di dati metagenomici. I risultati ottenuti in questa parte del lavoro sottolineano questa necessità. In particolare, le performace di uno degli strumenti più usati per la classificazione tassonomica delle sequenze di DNA batterico codificanti le subunità ribosomiali (il ``Ribosomal Database Project'' Classifier) sono state analizzate. Questa analisi ha mostrato una diminuzione del grado di risoluzione dell'algoritmo, passando dal livello tassonomica di Phylum a quello di Genere; ponendo l'accento sulla necessità di sviluppare algoritmi nuovi e più affidabili appositamente designati per analisi ``meta''. Per venire in contro a questa necessità, è stato sviluppato un nuovo software bioinformatico in grado di processare le sequenze metagenomiche grezze. Questo tool consente di rimuovere le regioni a bassa qualità da un set di sequenze metagenomiche, riducendò al minimo la quantità di dati persi. In aggiunta, il software in questione riesce a rifinire un grande numero di sequence in modo seriale, utilizzando quindi una piccola quantità di memoria dipendente unicamente dalla lunghezza delle sequenze e non dal loro numero rendendolo particolarmente adatto ad analisi metagenomiche.\\
La seconda parte di questo lavoro è composta da tre capitoli chiamati: ``Bacteria-environment interaction'', ``A life in transition: bacteria in harsh environments'' e ``Bacteria in agro-environments''. In questa parte sono stati discussi sia aspetti tassonomici che funzionali di comunità batteriche in ambienti naturali diversi. Cambiamenti nella composizione delle comunità batteriche sono stati analizzati in tre tipologie di ambienti: ambienti naturali, ambienti estremi e ambienti agricoli. In tutte le tipologie ambientali studiate, le comunità batteriche sono risultate essere capaci di modificare la loro struttura sia in risposta ad agenti perturbanti (gradiente antropico, salinità, inquinamento...) sia nel tempo. Lo sviluppo e la messa a punto di metodiche per lo studio di comunità batteriche complesse (come ad esempio nel suolo) sta diventando una tappa forzata per l'analisi di dati metagenomici dove anche semplicemente ottenere una visione di insieme della comunità in studio può essere motlo difficoltoso.\\
Nell'unlitma parte del mio lavoro, è stato investigato il ruolo di consorzi batterici come ``ospiti'' di organismi pluricellulari, in particolare l'uomo ed alcuni animali. Dal lato animale, il micorbiota intestinale di talitridi (\textit{Amphipoda}) è stato confrontato sia in specie diverse che in individui della stessa specie provenienti da aree geografiche differenti. I risultati ottenuti hanno mostrato che le comunità intestinali di questi animali sono molto diversificate e sembrano essere dipendenti sia dalla dieta (a sua volta legata dall'area geografica in studio) sia dalla specie. Sul lato umano invece, la composizione del microbiota è stata analizzata in pazienti affetti da una precisa patologia: la fibrosi cistica. In patologie strettamente legate ad infezioni poli-microbiche (come la fibrosi cistica) è molto importante capire a fondo le dinamiche che intrcorrono all'interno delle comunità batteriche in modo da sviluppare nuove cure o per migliorare quelle esistenti. In aggiunta, i risultati riportati mostrano che la selezione particolari firme tassonomicghe all'interno delle popolazioni batteriche umane può essere legato all'insorgenza o meno di particolari stai infettivi (peggioramento delle condizioni dei pazienti affetti dalla patologia).

\end{otherlanguage}